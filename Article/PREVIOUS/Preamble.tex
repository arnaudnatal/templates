\usepackage[doi, natbibapa]{apacite}

\usepackage{enumitem}
\usepackage[english]{babel}

\usepackage[T1]{fontenc}
\usepackage[utf8]{inputenc}

\let\CheckCommand\providecommand
\usepackage{microtype}
\usepackage{hyperref}

\usepackage{lscape}
\usepackage{graphicx}
\usepackage{amssymb,amsmath}
\usepackage{url}
\usepackage{longtable}
\usepackage{tabu}
\usepackage{threeparttable} 
\usepackage{array}
\usepackage{booktabs}

\usepackage[font=small,skip=1em,labelfont=bf,justification=justified,singlelinecheck=false]{caption}

\usepackage[normalem]{ulem}

\usepackage{setspace}
\usepackage{fullpage}
\usepackage{eso-pic}

\usepackage[explicit, clearempty]{titlesec}
\usepackage[tableposition=top]{caption}
%\usepackage{titlesec}
\usepackage[a4paper]{geometry}

\usepackage{rotating}
\usepackage{hvfloat}
\usepackage{wrapfig}
\usepackage{tfrupee}  

\usepackage{fancyhdr}
\usepackage{lipsum}  
\usepackage{lastpage}

\usepackage{changepage}




% *****************************************************************
% Estout related things
% *****************************************************************
\usepackage{adjustbox}
\newcommand{\sym}[1]{\rlap{#1}}% Thanks to David Carlisle

\let\estinput=\input% define a new input command so that we can still flatten the document
%\let\estinput=\@@input

\newcommand{\estwide}[3]{
		\vspace{.75ex}{
			\begin{tabular*}
			{\textwidth}{@{\hskip\tabcolsep\extracolsep\fill}l*{#2}{#3}}
			\toprule
			\estinput #1
			\bottomrule
			\addlinespace[.75ex]
			\end{tabular*}
			}
		}	

\newcommand{\estauto}[3]{
		\vspace{.75ex}{
			\begin{adjustbox}{max width=\textwidth}
			\begin{tabular}{l*{#2}{#3}}
			\toprule
			\estinput #1
			\bottomrule
			\addlinespace[.75ex]
			\end{tabular}
			\end{adjustbox}
			}
		}

% Allow line breaks with \\ in specialcells
	\newcommand{\specialcell}[2][c]{%
	\begin{tabular}[#1]{@{}c@{}}#2\end{tabular}}

% *****************************************************************
% Custom subcaptions
% *****************************************************************
% Note/Source/Text after Tables
\newcommand{\figtext}[1]{
	\vspace{-1.9ex}
	\captionsetup{justification=justified,font=footnotesize,singlelinecheck=false}
	\caption*{\hspace{6pt}\hangindent=1.5em #1}
	}
\newcommand{\fignote}[1]{\figtext{\emph{Note:~}~#1}}

\newcommand{\figsource}[1]{\figtext{\emph{Source:~}~#1}}

% Add significance note with \starnote
\newcommand{\starnote}{\figtext{* p < 0.1, ** p < 0.05, *** p < 0.01. Standard errors in parentheses.}}

% *****************************************************************
% siunitx
% *****************************************************************
\usepackage{siunitx} % centering in tables
	\sisetup{
		detect-mode,
		tight-spacing		= true,
		group-digits		= false ,
		input-signs		= ,
		input-symbols		= ( ) [ ] - + *,
		input-open-uncertainty	= ,
		input-close-uncertainty	= ,
		table-align-text-post	= false
        }





% *****************************************************************
% Équations
% *****************************************************************
\usepackage{IEEEtrantools}


% *****************************************************************
% Annexes
% *****************************************************************
\usepackage[toc,page]{appendix}


% *****************************************************************
% Sources + notes
% *****************************************************************
\newcommand{\sourcefig}[1]{\vspace{-2em} \caption*{ \textbf{Source:} {#1}} }
\newcommand{\notefig}[1]{\vspace{-2em} \caption*{\footnotesize \textit{Note}: {\footnotesize #1}} }

\addto\captionsenglish{\renewcommand{\figurename}{\textbf{Figure}}}


% *****************************************************************
% Abstract
% *****************************************************************
\let\abstractname\abstracteng

% *****************************************************************
% Hypothèses
% *****************************************************************
\usepackage{ntheorem}
\theoremseparator{:}
\newtheorem{hyp}{Hypothesis}




% *****************************************************************
% Bigcenter
% *****************************************************************

\makeatletter
\newskip\@bigflushglue \@bigflushglue = -100pt plus 1fil
\def\bigcenter{\trivlist \bigcentering\item\relax}
\def\bigcentering{\let\\\@centercr\rightskip\@bigflushglue%
\leftskip\@bigflushglue
\parindent\z@\parfillskip\z@skip}
\def\endbigcenter{\endtrivlist}
\makeatother

% *****************************************************************
% Lignes de code
% *****************************************************************
\usepackage{listings}
\lstset{ 
basicstyle=\scriptsize\ttfamily,
breaklines=true,
keywordstyle=\bf \color{blue},
commentstyle=\color[gray]{0.5},
stringstyle=\color{red},
showstringspaces=false,
numbers=left,
numberstyle=\tiny \bf \color{blue},
stepnumber=1,
numbersep=10pt,
firstnumber=1,
numberfirstline=true,
frame=leftline,
xleftmargin=0.5cm
}
 
% *****************************************************************
% Auteurs en bleu
% *****************************************************************
\usepackage{xcolor}
\usepackage{colortbl}

\hypersetup{colorlinks,linkcolor={red},citecolor={teal},urlcolor={blue}}

\usepackage{color,soul}



% *****************************************************************
% Mail
% *****************************************************************
\newcommand{\email}[1]{\href{mailto:#1}{\nolinkurl{#1}}}


% *****************************************************************
% Résumé et mots clés
% *****************************************************************
 \newenvironment{resab}[1]
{\begin{adjustwidth}{0cm}{0cm} \hangafter =1\par
    {\normalsize\bfseries #1\ \\ }\normalsize}
{\end{adjustwidth}\medskip}

 \newenvironment{keywords}
{\begin{adjustwidth}{0cm}{0cm} \hangafter =1\par
    {\normalsize\itshape Keywords:}~\normalsize}
{\end{adjustwidth}\medskip}

 \newenvironment{jelcodes}
{\begin{adjustwidth}{0cm}{0cm} \hangafter =1\par
    {\normalsize\itshape JEL Codes:}~\normalsize}
{\end{adjustwidth}\medskip}


% *****************************************************************
% Encadré
% *****************************************************************
\usepackage{tikz}

% Thick
\def\checkmark{\tikz\fill[scale=0.4](0,.35) -- (.25,0) -- (1,.7) -- (.25,.15) -- cycle;} 

\newcommand{\titlebox}[2]{%
\tikzstyle{titlebox}=[rectangle,inner sep=10pt,inner ysep=10pt,draw]%
\tikzstyle{title}=[fill=white]%
%
\bigskip\noindent\begin{tikzpicture}
\node[titlebox] (box){%
    \begin{minipage}{0.94\textwidth}
#2
    \end{minipage}
};
%\draw (box.north west)--(box.north east);
\node[title] at (box.north) {#1};
\end{tikzpicture}\bigskip%
}

% *****************************************************************
% Changer la forme des titres
% *****************************************************************
 %\titleformat{\section}[block]			%section + style prédéfini par l'extension (block = 1 ligne)
 %{\sffamily\bfseries\LARGE\titlerule[1pt]}						%format pour le titre + label
 %{\sffamily\bfseries\LARGE}						%format pour le titre + label
 %{\sffamily\bfseries\LARGE\arabic{section}}		%format que pour le label
 %{0.5cm}								%espace qui sépare le label du titre
 %{#1}									%description du style pour le titre uniquement
 %\titlespacing{\section}				%section
 %{0cm}									%espace à gauche du titre
 %{3em}									%espace verticale AVANT le titre
 %{1em}									%espace verticale APRES le titre
 %%{0cm} 								%espace à droite du titre: mettre la même valeur que gauche pour un peu centrer

 %\titleformat{\subsection}{\sffamily\bfseries\Large}{\thesubsection}{0.4cm}{#1}
 %\titlespacing{\subsection}
 %{1cm}
 %{2em}
 %{1ex}

 %\titleformat{\subsubsection}{\sffamily\bfseries\large}{\thesubsubsection}{0.4cm}{#1}
 %\titlespacing{\subsubsection}
 %{2cm}
 %{2em}
 %{1ex}

% *****************************************************************
% Style de la date
% *****************************************************************
%\def\mydate{\leavevmode\hbox{\the\year-\twodigits\month}}
\def\mydate{\leavevmode\hbox{\the\month-\twodigits\year}}
\def\twodigits#1{\ifnum#1<10 0\fi\the#1}

% *****************************************************************
% En tête
% *****************************************************************

% Pour les numéros de pages
%\pagestyle{fancy}

% Si tu veux mettre les numéros genre: 1/22
% Il faut que tu écrives
% "\thepage/\pageref{LastPage}"
% Sans les " " dans les lignes en bas: \fancyhead[...

% Ca c'est pour enlever la barre horizontale sous l'entête
% Pour la laisser tu mets "1pt" au lieu de 0
\renewcommand{\headrulewidth}{0pt}
\renewcommand{\footrulewidth}{0pt}

% fancyhead pour l'entête
% fancyfoot pour le pied de page
% L=left; R=right; C=center
%\fancyhead[L]{\textcolor{gray}{\textsf{\textit{Revue de la littérature autour du mariage: le cas de l'Inde}}}}
%\fancyhead[R]{\textcolor{gray}{\textsf{Natal, A. (\mydate)}}}
%\fancyfoot[C]{\thepage}
%updmap.exe --admin

%\lhead{\textcolor{gray}{\textsf{\textit{Revue de la littérature autour du mariage: le cas de l'Inde}}}}
%\rhead{\textcolor{gray}{\textsf{Natal, A. (\mydate)}}}
\cfoot{\thepage}

% *****************************************************************
% Jatis
% *****************************************************************
\newcommand{\jati}[1]{\textit{j\={a}ti{#1}}}


% *****************************************************************
% À développer
% *****************************************************************
\newcommand\dev[1]{\textbf{\textcolor{red}{#1}}}

% *****************************************************************
% Fonts
% *****************************************************************
%\usepackage{tgbonum}
\usepackage{kpfonts}

% *****************************************************************
% Taille des tableaux
% *****************************************************************
\let\oldtabular=\tabular
\def\tabular{\small\oldtabular}
%\def\tabular{\normalsize\oldtabular}

% *****************************************************************
% Style de la biblio
% *****************************************************************
\bibliographystyle{apacite}

% *****************************************************************
% Numérotation
% *****************************************************************
%\usepackage{lineno}

% *****************************************************************
% Titre et page de garde
% *****************************************************************
\usepackage{titling}
\setlength{\droptitle}{-2cm}
\pretitle{\begin{center}\fontsize{24pt}{10pt}\selectfont\bfseries}
\posttitle{\par\end{center}\vskip 1ex}
\preauthor{\begin{center}
    \large \lineskip 0.5em}
\postauthor{\par\end{center}}
%\thanksheadextra{1,}{}
\thanksheadextra{}{}
\setlength\thanksmarkwidth{.5em}
\setlength\thanksmargin{-\thanksmarkwidth}


% *****************************************************************
% Symboles
% *****************************************************************

\def\@fnsymbol#1{\ensuremath{\ifcase#1\or *\or \dagger\or \ddagger\or
   \mathsection\or \mathparagraph\or \|\or **\or \dagger\dagger
   \or \ddagger\ddagger \else\@ctrerr\fi}}
   
\makeatletter
\newcommand{\ssymbol}[1]{^{\@fnsymbol{#1}}}
\makeatother


% *****************************************************************
% Tables and figures
% *****************************************************************
\newcommand\figaround[1]{%
\begin{center}
\textbf{[Figure \ref{#1} around here]}
\end{center}
}%

\newcommand\tabaround[1]{%
\begin{center}
\textbf{[Table \ref{#1} around here]}
\end{center}
}%


% *****************************************************************
% Clipboard
% *****************************************************************
\usepackage{clipboard}

%\newclipboard{basename}
%\openclipboard{basename}
%\Copy{key}{content}
%\Paste{key}


% *****************************************************************
% Icones
% *****************************************************************
\usepackage{fontawesome}
